\documentclass[14pt,a4paper,twoside]{extarticle}
\usepackage{cmap}
\usepackage[T2A]{fontenc}
\usepackage[cp1251]{inputenc}
\usepackage[russian]{babel}
\usepackage{amsmath}
\usepackage{setspace}
\usepackage{cite}

\newcommand{\lquot}{``}
\newcommand{\rquot}{''}

%To change the layout of the itemize (no vertical spacing between)
\def\itemhook{%
  \setlength{\topsep}{8pt plus 3pt minus 3pt}%
  \setlength{\itemsep}{4pt plus 1pt minus 1pt}%
  \setlength{\partopsep}{0pt}%
  \setlength{\parsep}{0pt}}
  
\makeatletter
\def\itemize{%
  \ifnum \@itemdepth >\thr@@\@toodeep\else
    \advance\@itemdepth\@ne
    \edef\@itemitem{labelitem\romannumeral\the\@itemdepth}%
    \expandafter
    \list
      \csname\@itemitem\endcsname
      {\def\makelabel##1{\hss\llap{##1}}%
        \itemhook \csname itemhook\romannumeral\the\@itemdepth\endcsname}%
  \fi}
\makeatother

\oddsidemargin=0pt
\evensidemargin=0pt
\textwidth=165mm

\topmargin=2.4mm
\headheight=0mm
\headsep=0mm
\textheight=248mm

\sloppy

\begin{document}

\begin{titlepage}
\begin{center}
\textsc{САНКТ-ПЕТЕРБУРГСКИЙ ГОСУДАРСТВЕННЫЙ УНИВЕРСИТЕТ}\\
\end{center}
\vspace{1.5cm}
\begin{flushright}
{На правах рукописи}
\end{flushright}
\vspace{1.5cm}
\begin{center}
{Борменталь Галина Леопардовна}
\par
\vspace{2cm}
\textsc{ВЛИЯНИЕ ЛУННОГО СВЕТА НА РОСТ ТЕЛЕГРАФНЫХ СТОЛБОВ}
\par
\vspace{2cm}
{05.13.11 --- Математическое и программное обеспечение\\
вычислительных машин, комплексов и компьютерных сетей}
\par
\vspace{2cm}
{АВТОРЕФЕРАТ\\
диссертации на соискание ученой степени\\
кандидата физико-математических наук}
\end{center}
\par
\vspace{3.5cm}
\begin{center}
{Санкт-Петербург\\
1889}
\end{center}
\end{titlepage}

\thispagestyle{empty}
{Работа выполнена на кафедре БЕЗсистемного программирования ма\-те\-ма\-ти\-ко-ме\-ха\-ни\-чес\-ко\-го
факультета Санкт-Петербургского государственного университета.

\vspace{0.8cm}

\begin{tabbing}
Научный руководитель:\quad\quad\=доктор физико-математических наук,\\
\quad\>проф. ЛЬВОВ Гепард Тигранович\\
\>\\
Официальные оппоненты:\>доктор физико-математических наук,\\
\quad\>проф. ПРЕОБРАЖЕНСКИЙ Филипп Филиппович\\
\quad\>(ОАО \lquot НИИЧАВО\rquot, Москва)\\
\>\\
\>кандидат физико-математических наук,\\
\quad\>ЧУДАКУЛЛИ Наверн Орландович\\
\quad\>(Unseen University, Анк-Морпорк)\\
\>\\
Ведущая организация:\>Институт благородных девиц\\
\quad\>(Смоленск)\\
\>
\end{tabbing}


\vspace{0.8cm}
\noindent Защита диссертации состоится ``\underline{\phantom{332}}''\underline{\phantom{xxxxxxxxxx}} 1889 года
в \underline{\phantom{25}} часов на заседании совета Д212.232.51 по защите
докторских и кандидатских диссертаций при Санкт-Петербургском государственном университете по адресу: 198504, Санкт-Петербург, Петродворец, Университетский пр., д.~28, математико-механический факультет, ауд. 3В.

\vspace{0.8cm}
\noindent С диссертацией можно ознакомиться в Научной библиотеке Санкт-Пе\-тер\-бург\-ского государственного
университета по адресу: 199034, Санкт-Петербург, Университетская наб., д.~7/9.

\vspace{0.8cm}
Автореферат разослан ``\underline{\phantom{332}}''\underline{\phantom{xxxxxxxxxx}} 1889 года.

\vspace{2.5cm}
\noindent Ученый секретарь\\
диссертационного совета\\
доктор физико-математических наук,\\
профессор\phantom{xxxxxxxxxxxxxxxxxxxxxxxxxxxxxxxxxxxxxx}Перфокартов. И. К.
}
\end{document}

