\begin{titlepage}
\begin{center}
\textsc{}\\
\end{center}
\vspace{1.5cm}
\begin{flushright}
{\it На правах рукописи}
\end{flushright}
\vspace{3.5cm}
\begin{center}
{Понизовкин Денис Михайлович}
\par
\vspace{2cm}
\textsc{
МАТЕМАТИЧЕСКАЯ МОДЕЛЬ РЕКОМЕНДАТЕЛЬНОЙ СИСТЕМЫ НА НЕЧЕТКИХ МНОЖЕСТВАХ КАК
	ЭФФЕКТИВНОЕ РАСШИРЕНИЕ АНАМНЕСТИЧЕСКИХ КОЛЛАБОРАТИВНЫХ МОДЕЛЕЙ
	}
\par
\vspace{2cm}
{05.13.17 --- Теоретические основы информатики}
\par
\vspace{2cm}
{Автореферат\\
диссертации на соискание ученой степени\\
кандидата технических наук}
\end{center}
\par
\vspace{3.5cm}
\begin{center}
%{Нижний Новгород\\
%2016}
\end{center}
\end{titlepage}

%% \thispagestyle{empty}
%% {Работа выполнена на кафедре БЕЗсистемного программирования ма\-те\-ма\-ти\-ко-ме\-ха\-ни\-чес\-ко\-го
%% факультета Санкт-Петербургского государственного университета.

%% \vspace{0.8cm}

%% \begin{tabbing}
%% Научный руководитель:\quad\quad\=доктор физико-математических наук,\\
%% \quad\>проф. ЛЬВОВ Гепард Тигранович\\
%% \>\\
%% Официальные оппоненты:\>доктор физико-математических наук,\\
%% \quad\>проф. ПРЕОБРАЖЕНСКИЙ Филипп Филиппович\\
%% \quad\>(ОАО \lquot НИИЧАВО\rquot, Москва)\\
%% \>\\
%% \>кандидат физико-математических наук,\\
%% \quad\>ЧУДАКУЛЛИ Наверн Орландович\\
%% \quad\>(Unseen University, Анк-Морпорк)\\
%% \>\\
%% Ведущая организация:\>Институт благородных девиц\\
%% \quad\>(Смоленск)\\
%% \>
%% \end{tabbing}


%% \vspace{0.8cm}
%% \noindent Защита диссертации состоится ``\underline{\phantom{332}}''\underline{\phantom{xxxxxxxxxx}} 1889 года
%% в \underline{\phantom{25}} часов на заседании совета Д212.232.51 по защите
%% докторских и кандидатских диссертаций при Санкт-Петербургском государственном университете по адресу: 198504, Санкт-Петербург, Петродворец, Университетский пр., д.~28, математико-механический факультет, ауд. 3В.

%% \vspace{0.8cm}
%% \noindent С диссертацией можно ознакомиться в Научной библиотеке Санкт-Пе\-тер\-бург\-ского государственного
%% университета по адресу: 199034, Санкт-Петербург, Университетская наб., д.~7/9.

%% \vspace{0.8cm}
%% Автореферат разослан ``\underline{\phantom{332}}''\underline{\phantom{xxxxxxxxxx}} 1889 года.

%% \vspace{2.5cm}
%% \noindent Ученый секретарь\\
%% диссертационного совета\\
%% доктор физико-математических наук,\\
%% профессор\phantom{xxxxxxxxxxxxxxxxxxxxxxxxxxxxxxxxxxxxxx}Перфокартов. И. К.
%% }
%% \end{document}

