
{\underline {\bf Во второй главе}} проведен анализ АКМ.
Множеством характеристик пользователей АКМ является
множество объектов $I$, вес характеристики --- это значение $\rho(u, i)$.
Если мощность множества $P_0$ мала, то правила вывода АКМ
не могут быть применены.
С малой мощностью данных связаны такие известные проблемы АКМ, как
разреженность матрицы (data sparsity) и холодный старт.

Реальные исходные данные обладают свойствами динамики и
неоднородности.
Свойство динамики заключается в том, что множество исходных данных
меняется во времени, так как меняются предпочтения пользователей, и
мощность множеств $U$, $I$ растет.
Пусть $u_a \ru u \text{ для } P_0$, но в силу динамики возможна ситуация, когда
$\rhu(u_a, u) > \varepsilon_p \text{ для } P_{\bot}$. Тогда
утверждение СОМ (\ref{srs-assert}) и, следовательно, правило $\Pi_C$
(\ref{srs-pi}) ложны в общем случае для любых исходных данных.

Свойство неоднородности заключается в том, что пользователи предпочитают
различные, не обязательно близкие по характеристикам, объекты, то есть
предпочтения пользователей не однородны:
$\forall i, j:  (u_a \R i) \wedge (u_a \R j) \not \Rightarrow (i \rt j)$.
Тогда $\forall i, j:  (u_a \R i) \wedge (i \rt k) \not \Rightarrow u_a \R k$,
то есть утверждение ООМ (\ref{ors-assert}) и, следовательно, правило $\Pi_O$
(\ref{ors-pi}) ложны в общем случае для любых исходных данных.

Таким образом, если данные обладают свойством неоднородности или динамики,
то СОМ или ООМ не гарантируют получение эффективного решения по критерию
качества решения в общем случае, поэтому АКМ не эффективны по критерию
стабильности.

Пусть правила $\Pi_C$ и $\Pi_O$
истинны (то есть выполняются эвристические утверждения). Рассмотрим условия,
которые влияют на качество решения.

В исследовании было определено достаточные условие,
при выполнении которых СОМ и ООМ гарантируют получение эффективного
решения задачи $pred$ по критерию качества.

Достаточным условием получения эффективного решения задачи $pred$
в СОМ является выполнение транзитивности отношения близости на
кластере соседей: $\forall u_1, u_2 \in \mathcal{N}_U: (u_1 \ru u_a) \wedge
(u_2 \ru u_a) \Rightarrow u_1 \ru u_2$.

Достаточным условием получения качественного решения задачи $topN$ в ООМ
является выполнение транзитивности отношения близости на
объединении обучающего, тестового и результирующего множеств:
$(i \rt j) \wedge (i \rt k) \Rightarrow (j \rt k), i, j, k \in I^a_0 \bigcup
I_{topN} \bigcup I_{\bot}, I_{\bot} = \{i_{\bot}, I_{0} = \{i_{0}\}\}$.

Выполнение достаточных условий зависит от того, какая
функция используется в качестве меры близости и какой пороговый
параметр ($\varepsilon_{i}$ или $\varepsilon_p$) этой функции
установлен для выявления выполнения отношения близости.
Пусть $\rhi$ является функцией косинуса угла между контентами, которые
представляются в виде векторов в ООМ, и $\varepsilon_i = 0,49$, тогда
транзитивность отношения $\rt$ не гарантируется; коэффициент корреляции Пирсона,
являющийся традиционной мерой близости СОМ, не обладает свойством
транзитивности. Разработчики РС должны учитывать выполнение достаточных условий
при подборе функции и ее порогового значения, однако не всегда возможно
подобрать эти параметры так, чтобы выполнялись достаточные условия, и РС
удовлетворяла требованиям заказчика. В общем случае АКМ не являются
эффективными моделями по критерию качества.

%\begin{itemize}
%	\item Точность:
%
%\item Точность по списку длины $L$: $\frac{1}{L} \sum \limits_{\rh(u_a, i) \in
%	\overline{P}_{\bot}}s(i)$, $L = 1..N$;
%
%\item Средняя точность:
%	$AveP = \frac{1}{1 + \sum \limits_{n=1}^{N} s(i_n)} \cdot \sum
%	\limits_{L=1}^{N} P@L$;
%
%\item
%  $NDCG = 1 - \frac{DCG}{IDCG}$, где $DCG = s(i_1) + \sum \limits_{n=2}^N
%		\frac{s(i_n)}{log_2(i_n)}$, где
%	$IDCG$ --- идеальное $DCG$, для которого $\forall$ $n=1..N$: $s(i_n) = 1$.
%
%\end{itemize}
%Рассмотрим проблемы, связанные с расчетом оценки
%$\mathcal{E}_{topN}$ при решении задачи $topN$
%на примере точности $P$. Очевидно, что вид точности при ее определении активным
%пользователем имел бы следующий вид:\\
%$\frac{1}{N} \sum \limits_{i \in I_{topN}}s(i),
%\begin{cases}
%	s(i) = 0& \text{если $u_a \R i$}\\
%	s(i) = 1& \text{иначе}
%\end{cases}
%$.\\
%С помощью такой оценки пользователь мог бы вычислить
%точность сформированного решения $I_{topN}$.
%Однако ООМ определяет отношение $u_a \R i$ на основании эвристического
%утверждения, поэтому для ООМ определении точности, как и решение,
%основано на правиле вычисления \ref{ors-assert}:\\
%$\mathcal{E}_{topN} = \frac{1}{N} \sum \limits_{i \in I_{topN}}s(i),
%\begin{cases}
%	s(i) = 0& \text{если $\exists$ $i_{\bot}: i \rt i_{\bot}$}\\
%s(i) = 1& \text{иначе}
%\end{cases}
%$.\\
%Пусть $I_{topN} = \{i: \exists i_{\bot} | i \rt i_{\bot}\}$. Тогда
%$\mathcal{E}_{topN} \le \varepsilon_0$, то есть решение эффективно по критерию
%качества.
%И, правда, по исходным данным задачи $topN$ верно, что $u_a \R i_{\bot}$, решение
%строится таким образом, что $i \rt i_{\bot}$. Поэтому, если правило вычисления
%истинно (!!! не правило истинно), то $u_a \R i$, что и требуется по задаче.
%Однако, если исходные данные обладают свойством неоднородности, то правило
%вычисления может быть ложным, поэтому отношение $u_a \R i$ может не выполняться.
%Таким образом, решение может состоять из объектов, для которых
%$\rho(u_a, i) > \varepsilon_0$, но при этом
%$\mathcal{E}_{topN} \le \varepsilon_0$. Тогда значение $\mathcal{E}_{topN}$
%не является объективным показателем.
%Для того, чтобы значение оценки задачи $topN$ являлось объективным
%показателем достаточно, чтобы правило $\Pi_{OOM}$
%было истинно.

Вычислительную сложность будем характеризовать асимптотической сложностью
алгоритмов решений. Общее и основное действие, которое производится при
решении задач в ООМ и СОМ
--- вычисление меры близости. Поэтому примем данную операцию элементарной
и вычислим асимптотическую сложность алгоритмов относительно этой операции.
Для решения задачи $topN$ строится матрица, элементами которой являются
значения мер близости между объектами, поэтому
асимптотическая сложность равна $O(n^2)$.
Для решения задачи $pred$ необходимо построить множество соседей,
поэтому асимптотическая сложность равна $O(|U|)$.
Реальные системы, к примеру, Amazon, работают с огромным числом пользователей
(свыше 29 миллионов) и объектов (свыше миллиона), поэтому
асимптотическая сложность алгоритмов решений СОМ и ООМ велика в условиях работы
с подобными массивами данных, которые характерны для реальных данных, поэтому
АКМ не являются эффективными по критерию вычислительной сложности.

В общем случае АКМ не являются эффективными РС по критериям вычислительной
сложности, качества решения и стабильности.
