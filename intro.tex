
{\bf Актуальность работы и степень разработанности.}
С интенсивным развитием веб-технологий, огромным и постоянно растущим числом
информации, доступной через интернет с помощью множества различных устройств,
популярными становятся рекомендательные системы (далее РС), которые облегчают
пользователю задачу поиска нужной информации путем рекомендации такой информации
или путем определения степени близости конкретной информации пользователю.

Исходные данные РС --- множество
$P := \{(u, i, \rho(u, i)): \rho(u, i) \ne \bot\}$,
где символ $\bot$ означает неизвестное значение, и:
\begin{itemize}
	\item $u \in U := \{1,...,m\}$ --- идентификаторы пользователей РС;

	\item $i \in I := \{1,...,n\}$ --- идентификаторы объектов предметной
области РС. Например, объектом может фильм кинематографической РС.
Для простоты изложения не будем каждый раз употреблять выражение
<<идентификатор пользователя или объекта>>, а будем обозначать коротко
<<пользователь или объект>>;

	\item $\rho: U \times I \rightarrow \{[0,1] \bigcup \bot\}$ --- функция оценки близости
		пользователей и объектов. Значение $\rho(u,i)$ показывает, насколько
		объект $i$ по своим характеристикам близок предпочтениям пользователя $u$.
		Как правило, оценки близости задаются самими пользователями во время
		работы с РС.
		Будем считать, что чем меньше значение оценки, тем объект ближе.
		Будем говорить, что между пользователем $u$ и
		объектом $i$ выполняется отношение близости $\R$, если
		$\rho(u, i) \le \varepsilon_{\R}$, где $\varepsilon_{\R} \in [0,1]$ --- некоторая
		малая фиксированная величина.
		Будем называть таких пользователей и объектов близкими.
	\end{itemize}

Зачастую в исследованиях РС исходные данные
представляются в виде матрицы, элементами которой являются
значения $\rho(u, i)$. Как правило, если $\rho(u, i) \ne \bot$,
то это значение задал сам пользователь за время работы с системой
(к примеру, поставил оценку фильму):

\begin{center}
$\mathcal{M}_{\rho} =
\begin{pmatrix}
	\rho(1,1)& ... & \rho(1,n)  \\
	...      & \bot & ...  \\
	\rho(m,1)& ... & \rho(m,n)  \\
\end{pmatrix}$.\\
\end{center}

Матрица
$\mathcal{M}_{\rho}$ является разреженной, то есть большинство
значений \\ $\rho(u, i) = \bot$.
Разреженность матрицы $\mathcal{M}_{\rho}$ является причиной
существования проблемы поиска нужной информации, иначе от системы требовалось
бы только выбрать подмножества объектов с заданной оценкой близости. Эта же
причина послужила толчком для
возникновения и развития РС как инструмента, способного снизить степень
разреженности для каждого пользователя (называемого в таком случае активным
и обозначаемого символом $u_a$) путем решения следующих двух задач:
\begin{enumerate}
	\item прогнозирования (обозначим как $pred$). По данной задаче требуется
		спрогнозировать неизвестное значение $\rho(u_a, i_{\bot})$.
		Спрогнозированное значение будем обозначать символов $\rh(u_a,
		i_{\bot})$.

\item $topN$. По данной задаче требуется сформировать подмножество объектов
	$I_{topN} = \{i: (u_a \R i) \wedge \rho(u_a, i) = \bot\}
		\wedge |I_{topN}| = N \le n$.
		%Так как неизвестно, выполняется
		%ли отношение $u_a \R i$ в силу того, что $\rho(u_a, i) = \bot$,
		%то выполнение отношения $u_a \R i$ определяется по значению
		%прогнозной функции: $u_a \R i \Leftrightarrow \rh(u_a, i) <
		%\varepsilon_{\R}$.
\end{enumerate}

%Разнообразие существующих оценок приводит к проблеме выбора той или иной
%оценки для определения качества полученных результатов. Некоторые исследователи
%вводят свои собственные показатели качества. Проблематично сравнивать
%результаты различных исследований и невозможно сравнить результаты решений
%задачи $topN$ и $p$.
%Отсутствие стандартизации оценок приводит к проблеме интеграции знаний в области
%РС и наносит ущерб их прогрессу. Данные проблемы были определены Дж. Херлокером.

Существуют различные математические модели РС,
которые задают способ
представления данных о пользователях и объектах и
методы решения задач.
В диссертации исследуется одна из самых известных,
хорошо изученных
	\footnote{Recommender Systems in Computer Science and Information Systems
	--- A Landscape of Research / D. Jannach, M. Zanker, M. Ge [и др.]
	// International Conference on Electronic Commerce and Web Technologies.
	2012. С. 76–87.}
моделей РС --- анамнестическая коллаборативная модель
РС (далее АКМ). АКМ является одним из классов
коллаборативных моделей, применяющих методы коллаборативной фильтрации.
Методы решений задач АКМ используют анамнестические алгоритмы, которые
основаны на аксиомах, являющихся эвристическимb утверждениями
	\footnote{
		Toward the next generation of recommender systems: A survey of the
		state-of-the-art and possible extensions / G Adomavicius, A Tuzhilin //
		IEEE transactions on knowledge and data engineering. Т. 17, С 734-749

		Umyarov A., Tuzhilin A. Improving Collaborative Filtering Recommendations
		Using External Data // ICDM ’08 Proceedings of the 2008 Eighth IEEE
		International Conference on Data Mining. 2008. С. 618–627.

		Wang Jun. Unifying user-based and item-based collaborative filtering approaches
		by similarity fusion // SIGIR ’06 Proceedings of the 29th annual international
		ACM SIGIR. 2006.

		Berkovsky S., Kuflik T., Ricci F. Cross-Domain Mediation in Collaborative
		Filtering // Proceedings of the 11th international conference on User Modeling.
		2007. С. 355–359.
	}.
РС, использующие АКМ, внедрены во многие известные веб-сервисы:
Amazon, Netflix, IMDB, Kinopoisk, LastFm и т.д.
Изучением коллаборативной фильтрации занимались такие известные исследователи как
Г. Адамовичус, Дж. Констан, Дж. Карипис, Г. Ф. Рикки, Г. Ву, Л. Рокач, Б. Сарвар, А. Тужилин, Б. Шапираи, Дж. Херлокер и др.
Несмотря на успешность, популярность, заявляемую разработанность методов
коллаборативной фильтрации, на то,
что эти методы были интегрированы в бизнес более пятнадцати лет назад,
и на то, что они
уже стали называться Г. Адамовичиусом, А. Тужилиным, М. Экстрандом,
Дж. Ридлом и Дж. Констаном традиционными,
существует ряд открытых актуальных проблем, связанных с их применением.

% Чтоб не докопались
% Из Recommender System for an IPTV Service Provider
% Content-based systems [1, 3, 21] recommend items similar to those that a user
% liked in the past, by considering their features 

% item based
%Item-Based Top-N Recommendation Algorithms
%The primary motivation behind these algorithms is the fact
%that a customer is more likely to purchase items that are similar to the items
%that he/she has already purchased in the past; thus, by analyzing historical
%purchasing information (as represented in the user–item matrix) we can automatically identify these sets of similar items and use them to form the top-N
%recommendations. %
%
% model based
%Evaluation of item-based top-N recommendation algorithms 
%In this paper we present one such class of model-based
%top-N
%recommendation algorithms. These algorithms first
%determine the similarities between the various items and then used them to identify the set of items to be recommended.%
%
%

%Одной из проблем является отсутствие единых теории, терминологии и
%обозначений АКМ. К примеру, один и тот же метод, заключающийся в фильтрации
%объектов, носит различные наименования: объектно-ориентирован-ный (например, в
%работе <<Item-Based Top-N Recommendation Algorithms>>)
%модельный (например, в работе <<Evaluation of item-based top-N recommenda-tion algorithms>>),
%контентный (например, в работе <<Recommender System for an IPTV Service
%Provider>>).

%Существующие исследования носят, в основном, эмпирический характер,
%и анализ причин, почему один и тот же метод хорошо работает на одних данных, а
%на других --- нет, не проводится.

Для того, чтобы описать существующие проблемы,
введем понятие эффективности модели.
Будем говорить, что модель РС эффективна по некоторому критерию, если она
удовлетворяет ему независимо от дополнительных условий и ограничений.
В работе рассматриваются три критерия эффективности РС: (1) качество
решения, (2) вычислительная сложность и (3) стабильность, где стабильностью
будем называть свойство системы решать задачу качественно независимо
от исходных данных.

Чтобы определить эффективность по критерию качества или стабильности
проводится тестирование. Для этого исходное множество данных $P$
разбивается на обучающее и тестовое множества, которые обозначим символами
$P_0$ и $P_{\bot}$ соответственно.
Если $(u, i, \rho(u, i)) \in P_0$, то будем обозначать такие объекты $i_0$.
Если $(u, i, \rho(u, i)) \in P_{\bot}$, то будем обозначать такие объекты $i_{\bot}$.
После получения результирующего множества
$\overline{P}_{\bot} = \{(u, i, \rh(u_a, i))\}$ в ходе решения задачи
по данным обучающего множества, проводится
сравнение результирующего множества с тестовым. Сравнение производится
с помощью функций, называемых оценками качества. Для каждой задачи существует
своя группа оценок, в которую входит некоторое число функций. Например,
некоторые из оценок задачи $pred$ --- это MAE, NMAE, RMSE,
из оценок задачи $topN$ --- точность P, точность P@L по списку длины L, средняя точность
P@L, NDCG.
Будем говорить, что решение задачи $t$ эффективно по критерию качества,
если $\mathcal{E}_{t}(\overline{P}_{\bot}, P_{\bot}) \le \varepsilon_{t},$ где
$t \in \{topN, pred\}$, $\mathcal{E}_{t}$ --- оценка качества
решения задачи $t$, $\varepsilon_{t} \in [0,1]$ --- некоторая фиксированная
величина.
%В диссертации исследуются используемые оценки решения и двумерные
%РС, а, точнее, их самые известные представители --- коллаборативные РС (далее
%КМ). Эти системы основаны на коллаборативной фильтрации.
Как говорилось выше, АКМ основаны на эвристических утверждениях.
Выполнение эвристических утверждений в существующих исследования
не рассматривается, но оно обусловлено свойствами исходных данных.
Если исходные данные обладают свойствами динамики или неоднородности, то
эвристические утверждения не выполняются. А, так как на них базируются методы
решений задач, то нет гарантии получения эффективного решения по критерию
качества на любых исходных данных, или на одних
и тех же данных, но меняющихся со временем. Поэтому коллаборативные модели
не являются эффективными по критерию стабильности. Реальные данные, как
правило, обладают и тем и другим свойством. Зависимость качества решения
от свойств исходных данных приводит к тому,
что конкретная реализация РС эффективна по критерию качества на определенных
данных или в определенный момент времени,
но может быть неэффективной на других данных или в следующий момент времени.
Поэтому не существует более или менее универсальной модели, которую можно
применять в различных предметных областях, а существующие модели с течением
времени требуют внесения доработок.
%и требует доработки и дополнительной настройки,
%для чего необходимо выделять дополнительные ресурсы.

%На эвристических утверждениях базируется оценка задачи
%$topN$, поэтому, если исходные данные обладают свойством динамики или
%неоднородности, то значение оценки $\mathcal{E}_{topN}$ не является объективным
%показателем. Поэтому становится невозможным объективно оценивать те или иные
%подходы, применяемые в решении задачи $topN$.

Если исходные данные таковы, что эвристические утверждения выполняются,
то получение эффективного решения по критерию качества
ограничено дополнительными условиями, зависящими от применяемых
мер сходства, выбор которых производится разработчиками АКМ.
В общем же случае дополнительные условия не выполняются, поэтому АКМ
не эффективны по критерию качества.

Алгоритмы решения задач АКМ имеют высокую асимптотическую сложность.
Поэтому АКМ не являются эффективными по критерию вычислительной сложности, и
существует проблема масштабируемости.

Актуальность работы обусловлена тем, что традиционная модель РС (то есть АКМ)
не является эффективной. Интерес решения
поставленных проблем носит коммерческий характер,
так как РС используется в таких областях, как интернет-магазины, масс-медиа,
цитирование, банковские системы, системы менеджмента, реклама и т.п.,
где наличие эффективной по определенным критериям модели может увеличить
прибыль.
Интерес решения поставленных проблем носит не только коммерческий, но и
академический характер, так как научное направление РС тесно связано с такими
областями, как информационный поиск, машинное обучение и теория принятия
решений, ежегодно, начиная с 2007 проводятся международные конференции RecSys
под эгидой Ассоциации вычислительной техники ACM, на которых рассматриваются
проблемы и решения, связанные с РС.


%Существующие коллаборативные модели обладают рядом открытых задач и не проанализированных свойств моделей: 
%\begin{itemize}
%\item математические модели  решений задач в АКМ основаны на аксиомах, которые являются эвристическими 
%утверждениями, выполнение которых не анализируется и не гарантируется;
%\item существующие исследования предлагают новшества, результативность которых подтверждается эмпирически 
%на узком круге входных данных;
%\item задачи РС, в общем виде, заключаются в определении близости
%информационного объекта предметной мера близости.
%области к предпочтениям пользователя. Коллаборативные модели вычисляют близость на основании косвенных 
%признаков, а не за счет вычисления функции, определенной на парах объект, пользователь.
%%, что может быть вычислено, к примеру, 
%%с помощью расстояния между пользователем и объектом. Однако коллаборативные модели определяют такую близость косвенно
%%на основании информации о близости либо объектов, либо пользователей и истории о действиях пользователя.
%%Функция расстояния между пользователем и объектом не определяется.
%\item условия, при которых коллаборативные модели гарантируют эффективное решение задач, неизвестны;
%\item не анализируется объективность показателя используемых оценок эффективности по критерию 1f
%\item отсутствует стандарт оценки эффективности решения по критерию 1), который мог бы быть применен 
%для установления эффективности решения задач, приводимых в различных исследованиях на различных входных данных;
%\item существует проблема масштабируемости.
%\end{itemize}
%% Решение перечисленных проблем приводит к получению более эффективной модели, которая может быть успешно использована в условиях динамики и 
%% неоднородности данных и в не зависимости от области приложения. 


%% перечисленных проблем имеет как академический, так и коммерческий интерес, так как 
%\hfill \break
{\bf Цель диссертационного исследования} --- разработать
формальную математическую модель РС, являющуюся эффективным расширением
АКМ (то есть разработанная модель должна позволять более эффективно
применять существующие коллаборативные методы), и определить в разработанной модели
методы решения задач более эффективные, чем коллаборативные.

Для достижения цели были поставлены следующие
{\bf основные задачи}:
\begin{itemize}
	\item ввести критерии оценки эффективности модели и, проведя теоретический
		анализ, показать, что
		в общем сдучае эффективность АКМ по введенным критериям
		зависит от ряда условий;
	\item разработать модель рекомендательной системы на нечетких
		множествах и определить такие алгоритмы решений задач в разработанной
		модели, эффективность которых по критериям качества, стабильности и
		вычислительной
		сложность будет выше алгоритмов коллаборативной модели для любых
		исходных данных.
		Провести теоретическое сравнение разработанной модели и
		коллаборативной по определенным критериям;
	\item разработать программное обеспечение, с помощью которого
		провести тестирование, в ходе которого решить задачи стандартными
		методами коллаборативной фильтрации и методами, определенными в
		нечеткой модели. Сравнить полученные результаты.
\end{itemize}

{\bf Научная новизна} полученных в диссертационной работе результатов:
\begin{enumerate}
\item впервые сформулированы достаточные условия, при выполнении которых гарантируется,
	что решения задач, полученные при применении АКМ,
	будут эффективными по критерию качества;
\item разработана оригинальная математическая модель РС, основанная не теории
	нечетких множеств. Разработанная модель
	является эффективным расширением АКМ и
	позволяет новым способом использовать доступную в современных
	условиях контекстную информацию о пользователях;
\item впервые определено отображение метаданных пользователя
	на множество метаданных объектов, которое используется для решения задач.
\end{enumerate}

{\bf Результаты, выносимые на защиту}:
\begin{itemize}
\item результаты анализа АКМ:
  \begin{itemize}
	\item достаточное условие, при выполнении которого гарантируется получение
		эффективного решения задачи $topN$ по критерию качества;
	\item достаточное условие, при выполнении которого гарантируется получение
		эффективного решения задачи $pred$ по критерию качества;
  \end{itemize}
\item математическая модель РС, являющаяся эффективным расширением АКМ;
%\item алгоритм решения задачи $pred$ методами коллаборативной
%	фильтрации в разработанной модели, для которого выполняется
%	выведенное достаточное условие;
\item методика формального задания взаимосвязи между информацией
	о пользователе и объекте,
	заключающийся в определении отображения пользователя на множество объектов,
	за счет которого обеспечивается вычисление прогнозной функции как
	расстояния между пользователем и объектом;
\item алгоритмы решения задач, основанные на использовании заданного
	расстояния между пользователем и объектом;
\item разработанное программное обеспечение, с помощью которого проводилось
	тестирование моделей, и сравнительный анализ полученных
	результатов тестирования.
\end{itemize}


{\bf Практическая значимость работы} заключается в использовании
полученных теоретических результатов для практической программной реализации
эффективной РС, которая может быть применена не только в таких стандартных
областях применения РС, как, например, интернет-магазины,
где известно предпочтение пользователя по отношению к объекту, но и
в областях, где можно определить более сложные взаимосвязи на основании
контекстной информации о пользователях и объектах \footnote{Adomavicius G., Tuzhilin A. Context-Aware Recommender SystemsUsing
collaborative filtering to weave an information // Recommender Systems
Handbook. 2011. С. 217–253.}.
Полученные результаты могут быть также использованы для повышения
эффективности уже существующей программной реализации АКМ.

{\bf Апробация результатов работы.}
Основные результаты диссертации докладывались и обсуждались на следующих научных мероприятиях:
\begin{enumerate}
\item XII международная конференция <<Russian Conference on \\Digital Libraries>>
	(г. Казань, 2010 г.);
\item Молодежная школа-семинар <<Модели и методы исследования систем
	структуры>>, (пос. Дивноморское, 2011, 2014);
\item Международная конференция <<Управление и оптимизация неголономных
	систем>>,(г. Переславль-Залесский, 2013);
\item Ученый совет кафедры прикладной математики и информатики
	НИУ ВШЭ (г. Нижний Новгород, 2016);
\item Ученый совет Исследовательского центра искусственного интеллекта
	ИПС им. А. К. Айламазяна РАН (г. Переславль-Залесский , 2016);
\item Семинар компании IT-Aces (г. Переславль-Залесский, 2016);
\item XIX международная конференция <<Data Analytics and Management in Data
	Intensive Domains>> (г. Москва, 2017)
\end{enumerate}

{\bf Публикации.} Основные научные результаты по теме диссертации опубликованы
в 3 печатных журналах, 2 из которых рекомендованы ВАК РФ, и 5 электронных
журналах, 3 из которых рекомендованы ВАК РФ и 1 включен в БД Scopus.

{\bf Личный вклад автора.} Автору принадлежат содержащиеся в
диссертации результаты теоретического исследования АКМ,
разработанная математическая модель на нечетких множествах,
алгоритмы решений задач, определенные в разработанной модели,
способы повышения эффективности коллаборативных методов в разработанной модели
и программное обеспечение для проведения тестирования.

{\bf Реализация и внедрение результатов работы.}
В ходе диссертационного
исследования было написано программное обеспечение на
языке программирования С++ для проведения тестирования и с возможностью
его дальнейшего применения в реальных проектах в качестве модуля,
обеспечивающего решение задачи $topN$ и $pred$ в различных моделях и
при применении различных алгоритмов. С помощью программного обеспечения
было проведено тестирование, на основании результатов которого
было проведено практическое сравнение разработанной модели и АКМ.

Было написано программное обеспечение, представляющее собой ядро
рекомендательного веб-сервиса, написанное на языке программирования Java.
Функциональность данного ядра заключается в решении задачи $topN$
методом, определенным в разработанной модели РС.

Данное программное обеспечение получило свидетельство
о государственной программы для ЭВМ – $\textnumero$ 2013612828 <<Контентный
рекомендательный сервис>>. На языке программировании Java были написаны
два демонстрационных веб-приложения, использующих ядро рекомендательного
веб-сервиса. Данные веб-приложения были представлены на семинаре компании <<IT-Aces>>.

Результаты исследования нашли применение в компании
<<IT-Aces>> в процессе проектирования внутренних веб-сервисов, что
подтверждается актом о внедрении.

{\bf Структура и объем диссертации}. Диссертация состоит из введения, пяти глав
и заключения, списка литературы и приложения. Материал изложен на 162
страницах, содержит 18 таблиц, 30 рисунков, 110 литературных источников и 2
приложений.

{\bf Во  введении} аргументирована  актуальность  темы  диссертационного
исследования, представлена степень ее проработанности, сформулированы цель и
задачи  исследования,  рассмотрены  объект,  предмет  и  методы  исследования,
отражены научная новизна, теоретическая и практическая значимость результатов,
приведены сведения о внедрении и использовании результатов.

{\bf В первой главе} заданы основные термины и обозначения, на основании
которых определены модель РС, модели АКМ, критерии эффективности
моделей, задачи, основные методы решений и способы оценки качества решений.

{\bf В второй главе} проведен анализ описанных в первой главе методов решений
и способов оценки. Определены основные проблемы, условия и ограничения АКМ.

{\bf В третьей главе} описана разработанная математическая модель на основе
нечетких множеств и методы решений задач в этой модели. Проведено теоретическое
сравнение разработанной модели с АКМ и показано, что разработанная модель
является эффективным расширением АКМ.


{\bf В четвертой главе} описано программное обеспечение, разработанное для
тестирования, множество входных данных, применяющееся для
тестирования, методы тестирования, результаты тестов. Проведен практический
сравнительный анализ разработанной модели и АКМ на основании полученных тестов.

{\bf В пятой главе} описано веб-приложение рекомендательного веб-сервиса,
алгоритмы которого основаны на применении разработанной модели.

%%%%%%%%%%%%%%%%%%%%%%%%%%%%
%%%%%%%%%%% ГЛАВА 1 %%%%%%%%
%%%%%%%%%%%%%%%%%%%%%%%%%%%%
\hfill \break

